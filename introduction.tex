% Thematische Einordnung: Ziele des Ganzen, Onion Routing, anonymisierte Kommunikation (grob!)
%Thomas
%-Worum geht es 
%	-Welches Paper 
%	-Wann geschrieben
%	-Welches Problem beschreibt es
%	-Auf welchen Ansätzen/Paper baut es auf

%\section{Introduction}
The paper \textit{Tor: The Second-Generation Onion Router} \cite{tor2004original} was written by Roger Dingledine, Nick Mathewson and Paul Syverson in 2004. 

It introduces a system for anonymous communication connections preventing traffic analysis or surveillance, which can be used to anonymize the traffic of a lot of real-world internet protocols like HTTP to request websites.


%It describes a circuit-based low-latency anonymous communication system called Tor. 
%Tor based on the concept of Onion Routing, which was first published in the paper \textit{Hiding routing information} in 1996 \cite{goldschlag1996hiding}. 
%Onion Routing is also a low latency and circuit based communication system.
%A low-latency network is a network that experiences small delay times \cite{latency}.
%Because of that Tor is able to anonymize interactive network traffic and can used in the world wide web. Circuit-based means that Tor establishes circuits across the network to communicate. 



In the first section \textit{related work} Tor is based on as well as other systems dealing with the same problem field of anonymous communication connections are presented. The following section describes how Tor works and what improvements Tor has implemented. The final section shows the influence the discussed paper had to science and the Tor network has to the internet and our society.