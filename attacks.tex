\section{Attacks and Defenses}
The original Tor paper includes a chapter, named \textit{Attacks and Defenses}, which lists a variety of attacks, and discusses how well Tor withstands them. It describes passive attacks, active attacks, directory attacks and attacks against rendezvous points. This chapter briefly describes the last two attacks. 

Directory attacks can be to destroy directory servers. If a few servers disappear the others still decide on a valid directory. If a directory server is subverted, the corrupt directory server can at worst case a tie-breaking vote to decide whether to include marginals ORs because ORs are included or excluded by majority vote. If a majority of direcory servers are subverted the attacker can include as many compromised ORs in the final directory as he wishes. 

Attacks against rendezvous points can be to make many introduction requests. An attacker could try do deny Bobs service by flooding his introduction points with requests. Because the introduction points can block requests that lack authorization tokens, it is possible to restrict the volume of requests. Furthermore an attacker could disrupt a location-hidden service by disabling its introduction points. This is not effective because the service can simply re-advertise itself at a different introduction point. By comprising an introduction point the attacker could flood, the hidden service with introduction requests or prevent valid requests. In the first case the provider of the hidden service is able to close the circuit. In the second case the provider could periodically test his introduction point to make sure that rendezvous requests receives them.