
\documentclass[11pt,twocolumn,a4paper,DIV=calc]{scrartcl}
\usepackage[protrusion=true,expansion=true]{microtype}

\usepackage[utf8]{inputenc}
\usepackage[T1]{fontenc}

\usepackage{graphicx}
\usepackage{hyperref}
\usepackage{url}
\usepackage{times}
\usepackage{helvet}


\begin{document}
%TBD: infos
\title{The title of the paper}
\author{anonymous authors}
\date{}
\maketitle


\section{Hints from etherpad}
	\subsection{Reading}

	\begin{itemize}
		\item read your paper multiple times
	   \item find out when and where it has been published
	   \item understand the relevant basics of the field
	   \item search for other papers on the same topic to judge the relevance and contribution of your paper
	     \item published at an earlier time or at a later time
	     \item dealing with other problems in the same field
	     \item dealing with similar problems in different fields
	     \item tackling a similar challenge or other challenges
	\end{itemize}

\subsection{Writing}

\begin{itemize}
	\item Write up a summary with your findings
   \item length restriction: fit it into 4 pages A4 max!
   \item two-column 11pt Times!
   \item please use LaTeX for the layout (see template below)

   \item You may (re-)use figures or formulas whenever appropriate (if referenced correctly)
   \item Do not re-use text from the paper (no copy\&paste)!

   \item upload your summary by the deadline to the skiconf tool
     \item authors remain anonymous => do NOT include your names!
     \item bonus: ensure that the PDF metadata does not disclose your login name etc.
   \item language: English
\end{itemize}

\subsection{General hints for summary}

\begin{itemize}
\item Briefly outline the "field" this paper belongs to, the challenges that exist in this field and how your paper fits into the bigger picture
 \item Explain the relevant basics and fundamentals
   \item What stuff do you have to know in order to understand the content of the paper and to judge its impact?
 \item Your summary should contain answers to the following:
   \item What are the problems to be solved? Why are they worthwhile to solve?
   \item What methods are used? How is the problem tackled?
   \item What results are obtained? What do the results mean?
 \item Additionally: outline the impact of the paper
   \item What is the "delta" that the authors of the paper achieved in comparison of the state of the art at the time the paper was published
   \item How is it different/better compared to related works?
     \item How has the paper been received by the scientific community?
     \item How do papers that have been published later talk about this paper?
     \item Is the delta of the paper of relevance today?
     \item Did the paper influence its field?
\end{itemize}

\subsection{General advice}
\begin{itemize}
 \item be as descriptive and concrete as necessary, but still as concise as possible
 \item give examples if suitable

 \item most important challenge
   \item your Summary should be *understandable on its own*
     \item you will have to leave out lots of (less relevant) details
     \item but: you may have to add things that are not described in the paper

  \item ask yourself: does the summary contain the most relevant contributions of the paper?

 \item do not confuse your summary with the "Abstract" at the beginning of a paper
   \item your summary contains more details than the abstract of the paper
   \item your summary contains information *about* the paper (which was not available when the paper was published)

  \item remember: do not copy text word by word from your paper (plagiarism)!!!

\end{itemize}

\newpage

\section{Introduction}

% Thematische Einordnung: Ziele des Ganzen, Onion Routing, anonymisierte Kommunikation (grob!)

% Thematische Einordnung: Ziele des Ganzen, Onion Routing, anonymisierte Kommunikation (grob!)
%Thomas
%-Worum geht es 
%	-Welches Paper 
%	-Wann geschrieben
%	-Welches Problem beschreibt es
%	-Auf welchen Ansätzen/Paper baut es auf

%\section{Introduction}
The paper \textit{Tor: The Second-Generation Onion Router} \cite{tor2004original} was written by Roger Dingledine, Nick Mathewson and Paul Syverson in 2004. 

It introduces a system for anonymous communication connections preventing traffic analysis or surveillance, which can be used to anonymize the traffic of a lot of real-world internet protocols like HTTP to request websites.


%It describes a circuit-based low-latency anonymous communication system called Tor. 
%Tor based on the concept of Onion Routing, which was first published in the paper \textit{Hiding routing information} in 1996 \cite{goldschlag1996hiding}. 
%Onion Routing is also a low latency and circuit based communication system.
%A low-latency network is a network that experiences small delay times \cite{latency}.
%Because of that Tor is able to anonymize interactive network traffic and can used in the world wide web. Circuit-based means that Tor establishes circuits across the network to communicate. 



In the first section \textit{related work} Tor is based on as well as other systems dealing with the same problem field of anonymous communication connections are presented. The following section describes how Tor works and what improvements Tor has implemented. The final section shows the influence the discussed paper had to science and the Tor network has to the internet and our society.

\section{Onion Routing/Technical background}

% Wie funktioniert das (theoretischer Überblick), 
% Paper aus Tor-Paper (41 Anonymous connections and onion routing, 48 Onion routing access configuration, 49 Towards an analysis of onion routing security, 27 Hiding routing information) 
% Einordnung in die Vorarbeiten (Bereits implementierte Anwendungen)

% Wie funktioniert das (theoretischer Überblick), 
% Paper aus Tor-Paper (41 Anonymous connections and onion routing, 48 Onion routing access configuration, 49 Towards an analysis of onion routing security, 27 Hiding routing information) 
% Einordnung in die Vorarbeiten (Bereits implementierte Anwendungen)
%Tom
%Onion Routing basierend auf den 4 Papern beschreiben
\section{Onion Routing/Technical background}
The goal of onion routing is not to provide anonymous communication. The goal is to limit the traffic analysis in that way to make it impossible to determine who is communication with whom through a public network. In the onion routing network an initiating application make connections through a sequence of machines called onion routers. Each onion router knows only its adjazent onion routers. Because of that an onion router is not able to know who is communication with whom. The data which is sent by the application is layerd like an onion.Each layer of the onion is encrypted by one onion router and contains the next hop in the route. \cite{reed1998anonymous}

\section{Tor}

% Inhalte des Papers (grob), Schwerpunkte (Verbesserungen in Tor gegenüber damals existierenden Systemen), Neuerungen in Tor, etc.
% Inhalte des Papers (grob), Schwerpunkte (Verbesserungen in Tor gegenüber damals existierenden Systemen), Neuerungen in Tor, etc.
%Mal schauen, eher beide...
%-Verbeserungen zu Onion routing:
%aus dem abstract: 
%by adding perfect forward secrecy, 
%congestion control, 
%directory servers,
%integrity checking, 
%configurable exit policies, 
%and a practical design for location-hidden services via rendezvous points.
\section{Tor}
Tor works on the real-world Internet and requires no special privileges or kernel modifications to increase the usability. In case of Tor the usability is a security requirement because it hide users among users and a system with fewer users provides less anonymity.
Additionally to onion routing, Tor added perfect forward secrecy, congestion control, directory servers, integrity checking, configurable exit policies and a practical design for location-hidden services via rendezvous points.

\textbf{Perfect forward secrecy:} In the original onion routing design a single hostile node can record traffic and later compromise succesive nodes in the circuit. To avoid this Tor uses an incremental path building design. Thant means, that the initiator negotiates session keys with each succecive node in the circuit. If these keys are deleted, the compromised nodes are not able to decrypt old traffic.

\textbf{Congestion control:} If many users choose the same OR-to-OR connectionin their circuits, that connection can be saturated. Without any congestion control this bottleneck can propagate through the whole network. To avoid this, Tor is able to control the circuits bandwidth in each OR. 

\textbf{Directory servers:} Both Onion routing and Tor has to send periodically state informations (known ORs with their current states) to their users. In the onion routing design they flood their network with these state informations. Tor uses so called \textit{directory servers} to distribute this informations through the network. The users download them periodically every 15 minutes via HTTP.

\textbf{Integrity checking:} Tor uses TLS between its nodes. Because of that an adversary is not able to change the content of the data.

\textbf{Rendezvous points:} If a user wants to connect with a hidden server the onion routing design uses long lived "reply onions", which are used to build a circuit to a hidden server. This is bad because the onion routing design does not provide forward security. For a connection between an user and a hidden server, tor uses so called rendezvouz points. If a user wants to create a connection to a hidden server, it uses the introduction points of this hidden server to send them a rendezvous point. The user and the hidden server uses this rendezvous point to create a circuit between them. Every hidden server has circuits to ORs, which are introduction points.\cite{rendezvousPoints}

\section{Importance/Impact}

% Heutige Verwendung, Größe des Netzwerks, auch Contra-Argumente: Nutzung durch Kriminelle etc., Referenzierungen des Papers

% Heutige Verwendung, Größe des Netzwerks, auch Contra-Argumente: Nutzung durch Kriminelle etc., Referenzierungen des Papers
%-Ein praktikables System
%-we feel that we must build a
%reputation for privacy, human rights, research, and other socially
%laudable activities. (Zitat aus unserem Paper Seite 14 links oben irgendwo)
%-Quelle: Internet (bei Berichten aus Krisenregionen, wie Kayro, Syrien), wenn es aus %em Darknet kommt.
%-Aktivisten können Anonym Bericht erstatten.
%-Kriminelle nutzen es auch (Zitat: Seite 14 oben links: Even though having more users would
%bolster our anonymity sets, we are not eager to attract the
%Kazaa or warez communities)


% also consider chapter 8
% current numbers (https://metrics.torproject.org/)

\section{Importance and impact}
A wide area network was briefly deployed after the invention of onion routing but it was not really used because it required a seperate "application proxy" for each supported application protocol and most of them were never written. Tor on the other hand uses the standard and near-ubiquitous SOCKS proxy interface, which supports most TCP-based programs without modification.

\begin{figure}
	\includegraphics[width=\columnwidth]{img/userstats.pdf}
	\caption{Estimated number of users connecting to Tor. Taken from \cite{tormetrics}.}
	\label{img_userstats}
\end{figure}

Because of that Tor is used in the world wide web since 2004 for anonymous connections. Individuals use Tor for censorship circumvention to reach othervise blocked content, to keep websites from tracking them or their family members, to publish services without needing to reveal the location of them or for sensitive communication. Tor is also used by companies or Journalists to communicate with activists or whistleblowers\cite{tor}. Currently there are about two million people using Tor as figure \ref{img_userstats} states. The peak in August if 2013 was probably caused by a botnet using Tor for the communication between the bots and the command and control server\cite{torblog_botnet}. This is an example for the downside of enabling anonymous connections. Tor can also be used by criminals to sell drugs, child porn, weapons or other illegal things.   

Apart from the practical impact of the Tor network the published paper also has a big impact on computer science and was cited more than 3000 times by other papers dealing with improvements for or attacks against Tor, analysis of practical effects of the Tor network or using the developed approach to enable anonymous connections in other fields like VoIP systems.

\end{document}


